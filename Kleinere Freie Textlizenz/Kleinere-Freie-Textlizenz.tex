\documentclass[a4paper,10pt]{scrartcl}
\usepackage{ulem}
\usepackage{pxfonts}
\usepackage{ngerman}
\usepackage[utf8]{inputenc}
\setkomafont{sectioning}{\rmfamily\bfseries\boldmath} %% roman (serif)

%opening
\title{Kleinere Freie Textlizenz v0.12}
\author{Arne Babenhauserheide}
\date{2006-11-21}

\begin{document}

\maketitle

\begin{abstract}

Eine freie Lizenz für Texte und andere Werke. Eine abgeschwächte GFDL. Sie erlaubt es, unter ihr lizensierte Werke in kommerzielle Werke einzubinden und verlangt dabei, dass der logische Abschnitt des unter ihr lizensierten Teiles frei bleibt. 

\end{abstract}

\section{Vorbemerkung}

Dokumente, die unter dieser Lizenz stehen können jederzeit zu einer "`GNU Lesser Free Document License"' jeglicher Version oder einer ähnlichen GNU Lizenz größer oder gleich Version 2 relizensiert werden, falls dadurch der Geist der Kleineren Freien Textlizenz in großen Teilen erhalten bleibt. 

Auch dürfen sie unter der GNU Simple Free Documentation License genutzt und zu ihr relizensiert werden. 

\section{Einleitung}

!! -Unfertige Version- !!

Dies ist eine Abschwächung der [http://www.gnu.org/copyleft/fdl.html GFDL], die ähnlich zur GFDL steht wie die LGPL zur GPL. Sie soll die folgenden vier grundlegenden Freiheiten für jeden Nutzer des Werkes gewährleisten: 

\begin{itemize}
 \item Die Freiheit das Werk zu jedem Zweck zu nutzen (Freiheit 0). 
 \item Die Freiheit das Werk an deine Bedürfnisse anzupassen (Freiheit 1). Zugriff auf die Quelldateien ist eine Vorbedingung dafür. 
 \item Die Freiheit Kopien des Werkes weiterzugeben, um Nachbarn und Freunden helfen zu können (Freiheit 2). 
 \item Die Freiheit das Werk zu verbessern und deine Verbesserungen zu veröffentlichen, so dass die gesamte Gemeinschaft davon profitiert (Freiheit 3). Auch hierfür ist Zugriff auf die Quelldateien eine Vorbedingung. 
\end{itemize}

Dokumente unter der Kleineren Freien Textlizenz dürfen jederzeit unter der GFDL ohne invariante Sektionen und Cover, oder unter der GSFDL relizensiert werden. 

Falls es eines Tages eine Lesser Variante der GFDL oder GSFDL geben sollte, dürfen Dokumente unter der KFT auch jederzeit unter diese Lesser GFDL ohne Invariante Sektionen und ohne Cover oder diese Lesser GSFDL relizensiert werden. 

Diese Lizenz wird noch verbessert, um die Problemstellung besser abzudecken. Daher kann ein Werk auch unter die "`Kleinere Freie Textlizenz v0.1 oder höher"' gestellt werden. Ein generisches "`KFT"' bedeutet, dass jede Version gewählt werden kann. Als höhere Version gilt nur, was den Geist der Lizenz erhält, der hauptsächlich durch die vier Freiheiten definiert wird. 

Die Kleinere Freie Textlizenz orientiert sich an der GNU LGPL aus der Bewegung für freie Software und bringt sie in den Bereich des geschriebenen Wortes. 

Jedes Werk, das unter der KFT liegt, muss entweder eine Kopie der KFT enthalten oder es muss eine KFT für den Nutzer direkt zugänglich sein (z.B. daneben ausgelegt oder auf des selben Webseite). 

\section{Grundlagen}

Diese Lizenz erlaubt die Nutzung und Veränderung des Textes zu jedem Zweck, solange die folgenden Bedingungen eingehalten werden: 

\begin{enumerate} 
\item Die Ursprungsautoren und alle weiteren Autoren müssen genannt werden. Mindestens 6 davon klar sichtbar oder Alle klar sichtbar. 

\item Alle Veränderungen an diesem Text und dem logischen Abschnitt zu dem er gehört, müssen unter dieser Lizenz freigegeben werden. 

\item Wenn er in ein größeres Gesamtwerk eingebunden ist, muss nur der Teil freigegeben werden, der logisch zu diesem Werk gehört, besonders jedoch der Teil der Quelldateien, der zu diesem logischen Abschnitt gehört. Beispiele folgen später. 

\item Wenn dieses Werk in ein designtes Dokument eingebunden wird, dann muss von dem Design nur der Teil freigegeben werden, der spezifisch zum logischen Abschnitt des Werks gehört, also z.B. eingebundene Bilder, die im Fließtext stehen, aber nur solche Dekorationen, die nur in diesem logischen Abschnitt verwendet werden. 

\item Wird der Text für ein abgeleitetes Werk verwendet (z.B. zu einem Lied gemacht), dann sind in dem logischen Bereich des Werks alle Zusätze eingeschlossen, die notwendig sind, um die Ableitung zu ermöglichen, allerdings nicht unbedingt das gesamte abgeleitete Werk. Bei einem Lied wären zum Beispiel Griffe, Noten und Text Teil des Werkes, die Aufnahme aber nicht. 
\end{enumerate}

\section{Definitionen}

\begin{enumerate} 
\item Die Veränderungen freigeben bedeutet, dass sie als Quellen zugänglich gemacht werden müssen. Diese Quellen müssen in einem freien Format vorliegen (z.B. reiner Text, vorzugsweise utf8). Wenn das Werk designt wurde, müssen zusätzlich die Teile der Designdateien freigegeben werden, die den logischen Abschnitt des freien Werkes enthalten (für Details siehe voriger Punkt). Die Quellen müssen nur denjenigen zugänglich gemacht werden, die auch das Werk erhalten, aber sie müssen die Möglichkeit haben, das Werk und die Quellen frei weiterzugeben, da beides unter dieser Lizenz steht. 

\item Zugänglich machen bedeutet, dass die Quellen bei einer Verbreitung des Werkes entweder beiliegen müssen, oder dem Werk ein mindestens 3 Jahre gültiges schriftliches Angebot beiliegen muss, die Quellen auf Anfrage zu liefern, und zwar zumindest auf dem Weg, auf dem das Werk weitergegeben wurde. Dafür darf maximal ein Betrag in Höhe der entstehenden Unkosten verlangt werden. Es dürfen auch zusätzliche alternative Wege angeboten werden. Werden weniger als 10 Kopien weitergegeben, wird diese Bedingung zu einem "`sollte"', damit auch eine kurze Weitergabe unter Freunden möglich ist, ohne gleich die Quellen zur Hand haben zu müssen. 
\end{enumerate}

\section{Bemerkungen}

\begin{itemize}
\item Richtschnur 1: Wenn das Werk für irgendetwas verwendet wird, muss auch jedem Nutzer die Möglichkeit gegeben werden, es für diesen Zweck zu verwenden, allerdings ohne dafür die Rechte an zusätzlichem, nicht zum logischen Abschnitt des freien Werkes gehörendem, Material zu geben. Ich verwende das Werk in einem Buch, also muss ich auch jedem anderen diese Möglichkeit eröffnen, solange der andere bereit ist, die von mir geschriebenen Teile, die nicht zum logischen Abschnitt des freien Werkes gehören, durch eigene zu ersetzen. 

\item Richtschnur 2: Jeder Nutzer muss die gleichen Nutzungsmöglichkeiten an dem logischen Abschnitt haben, zu dem das Werk gehört, mit Ausnahme der Urheberschaft an eigenen Beiträgen (jedoch mit Nutzungsrechten). Wenn ein Nutzer für sich eine weitere Nutzungsmöglichkeit findet, sie ermöglicht und das Ergebnis veröffentlichen will, dann muss er auch denjenigen, die das Ergebnis von ihm beziehen diese Möglichkeit einräumen.

\item Disclaimer: Ich bin kein Anwalt, und ich gebe keine Gewähr auf das korrekte "`funktionieren"' dieser Lizenz. 
\end{itemize}


Weitere Details folgen, sobald sie entworfen sind. Stabil sind die vier Freiheiten und der Sinn der Lizenz.




\end{document}
